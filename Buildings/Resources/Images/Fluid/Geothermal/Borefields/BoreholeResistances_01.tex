\documentclass{article}

\usepackage{circuitikz}
\usepackage{tikz}
\usepackage{nopageno}
\usepackage[active,tightpage]{preview}  %generates a tightly fitting border around the work
\PreviewEnvironment{tikzpicture}
\setlength\PreviewBorder{2mm}

\begin{document}
\usetikzlibrary{decorations.pathreplacing}
\usetikzlibrary{arrows}

\begin{tikzpicture}[scale=0.6,>=angle 60]
% Outline of borehole and pipes
\draw[line width=1pt] (0,2) circle(4);
\draw[line width=1pt] (-2.5,2) circle(0.7);
\draw[line width=1pt] (2.5,2) circle(0.7);
% Fluid and borehole wall nodes
\fill[red] (-0.1,-1.9) -- (-0.1,-2.1) -- (0.1,-2.1) -- (0.1,-1.9) -- cycle;
\fill[red] (-2.6,1.9) -- (-2.6,2.1) -- (-2.4,2.1) -- (-2.4,1.9) -- cycle;
\fill[red] (2.6,1.9) -- (2.6,2.1) -- (2.4,2.1) -- (2.4,1.9) -- cycle;
% Inner delta circuit of thermal resistances
\draw[line width=1pt,red] (-1.25,0) -- (-0.5,0) -- (-0.4,0.2) -- (-0.2,-0.2) -- (0,0.2) -- (0.2,-0.2) -- (0.4,0.2) -- (0.5,0) -- (1.25,0);
\draw[line width=1pt,red,cm={cos(58) ,-sin(58) ,sin(58) ,cos(58) ,(-0.625,-1)}] (-1.25,0) -- (-0.5,0) -- (-0.4,0.2) -- (-0.2,-0.2) -- (0,0.2) -- (0.2,-0.2) -- (0.4,0.2) -- (0.5,0) -- (1.25,0);
\draw[line width=1pt,red,cm={cos(58) ,sin(58) ,sin(58) ,-cos(58) ,(0.625,-1)}] (-1.25,0) -- (-0.5,0) -- (-0.4,0.2) -- (-0.2,-0.2) -- (0,0.2) -- (0.2,-0.2) -- (0.4,0.2) -- (0.5,0) -- (1.25,0);
% Fluid to grout thermal resistances
\draw[line width=1pt,red,cm={cos(58) ,-sin(58) ,sin(58) ,cos(58) ,(-1.85,1)}] (-1.25,0) -- (-0.5,0) -- (-0.4,0.2) -- (-0.2,-0.2) -- (0,0.2) -- (0.2,-0.2) -- (0.4,0.2) -- (0.5,0) -- (1.25,0);
\draw[line width=1pt,red,cm={cos(58) ,sin(58) ,sin(58) ,-cos(58) ,(1.85,1)}] (-1.25,0) -- (-0.5,0) -- (-0.4,0.2) -- (-0.2,-0.2) -- (0,0.2) -- (0.2,-0.2) -- (0.4,0.2) -- (0.5,0) -- (1.25,0);
% Grout node capacitance (left)
\draw[line width=1pt,red] (-1.25,0) -- (-1.85,-0.4);
\draw[line width=1pt,red] (-2.05,-0.1) -- (-1.65,-0.7);
\draw[line width=1pt,red] (-2.15,-0.18) -- (-1.75,-0.78);
% Grout node capacitance (right)
\draw[line width=1pt,red] (1.25,0) -- (1.85,-0.4);
\draw[line width=1pt,red] (2.05,-0.1) -- (1.65,-0.7);
\draw[line width=1pt,red] (2.15,-0.18) -- (1.75,-0.78);
% Borehole wall node annotation
\node[below] at (0,-2.1) {\large $T_b$};

% Arrow to lower figure
\draw[line width=2pt,->] (2.4,0) to [out=-50,in=65] (0.7,-4.7);

% Thermal resistances in series
\draw[line width=1pt,red,cm={1,0,0,1 ,(0.85,-6.5)}] (-1.25,0) -- (-0.5,0) -- (-0.4,0.2) -- (-0.2,-0.2) -- (0,0.2) -- (0.2,-0.2) -- (0.4,0.2) -- (0.5,0) -- (1.25,0);
\draw[line width=1pt,red,cm={1,0,0,1 ,(-1.65,-6.5)}] (-1.25,0) -- (-0.5,0) -- (-0.4,0.2) -- (-0.2,-0.2) -- (0,0.2) -- (0.2,-0.2) -- (0.4,0.2) -- (0.5,0) -- (1.25,0);
% Fluid and borehole wall nodes
\draw[line width=1pt,red,-o] (2.1,-6.5) -- (3,-6.5);
\draw[line width=1pt,red] (-3.05,-6.5) circle(0.15);
% Pipe section
\draw[line width=1pt] (0.1,-5) arc (150:210:3);
\draw[line width=1pt] (2.5,-5) arc (150:210:3);
\draw[line width=1pt] (0.1,-5) -- (2.5,-5);
\draw[line width=1pt] (0.1,-8) -- (2.5,-8);
% Grout node capacitance (lower)
\draw[line width=1pt,red] (-0.75,-6.5) -- (-0.75,-7);
\draw[line width=1pt,red] (-1.1,-7) -- (-0.4,-7);
\draw[line width=1pt,red] (-1.1,-7.15) -- (-0.4,-7.15);
\draw[line width=1pt,red] (-0.75,-7.15) -- (-0.75,-7.5);
\draw[line width=1pt,red] (-0.95,-7.5) -- (-0.55,-7.5);
\draw[line width=1pt,red] (-0.85,-7.6) -- (-0.65,-7.6);
\draw[line width=1pt,red] (-0.8,-7.7) -- (-0.7,-7.7);
% Text annotations
\node[right] at (3,-6.5) {\large Fluid};
\node[left] at (-3,-6.5) {\large $T_b$};
\node[above] at (1,-6.35) {\large Pipe};
\node[above] at (-1.65,-6.2) {\large Grout};
\end{tikzpicture}

\end{document}